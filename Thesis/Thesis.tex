\documentclass[german,version-2020-11]{uzl-thesis}


% Copy this file as a template for your thesis. You will have to take
% action at all places marked by
%
% !!!!!!!!!!!!!!!!!!!!!!!!!!!!!!!!!!
% !!! Your action is needed here !!!
% !!!!!!!!!!!!!!!!!!!!!!!!!!!!!!!!!!
%
% The first place your action is needed is the first line of this
% document:
%
%
% Language of the thesis:
%
% You must use either 'german' or 'english' above, depending on the
% language used in the main text. This will automatically setup a lot
% of things in the background.
%
%
% Version of the class:
%
% You must specify which version of the thesis class is to be
% used. This is important in case the class style changes in later
% years, but we still want an older thesis to look the same, even when
% things are changed in the class.
%
% Do not change or remove the version-xxxx key.
%
%
% Text encoding:
%
% Your thesis *must* be encoded in utf8 (unicode), which is the
% default in most editors these days. Do *not* change this to latin8.



%%%
%
% Main setup:
%
%%%
%
% You must use the \UzLThesisSetup command to specify numerous things
% about your thesis. This includes the entries on the title page, the 
% abstracts, and the bibliography style. You do so by specifying
% so-called "values" for so-called "keys". For instance, 
% for the key "Autor" you must provide your name as the value. You do
% so by writing 'Autor = {Max Mustermann}', that is, the value is put
% into curly braces. You can use the \UzLThesisSetup command
% repeatedly and the order in which you provide the keys is not
% important. 
%
% Everything shown on the title page must be in German -- even
% if the thesis is written in English! Just insert German text for
% German keys and English text for English keys (like 'Abstract' needs
% English text, while 'Zusammenfassung' needs German text).

\UzLThesisSetup{
  %
  % !!!!!!!!!!!!!!!!!!!!!!!!!!!!!!!!!!
  % !!! Your action is needed here !!!
  % !!!!!!!!!!!!!!!!!!!!!!!!!!!!!!!!!!
  %
  % First, specify the institut or clinic at which the thesis was
  % written. You get the logo file from them (make sure it has the
  % correct size, namely the same as the example). If they do not have
  % a logo, the university's default logo is used.
  %
  % The 'verfasst' gets two arguments. Change the first to {an der}
  % for clinics, as in 'Verfasst = {an der}{Medizinischen Klinik I}'
  %
  Logo-Dateiname        = {uzl-thesis-logo-itcs.pdf},
  Verfasst              = {am}{Institut für Theoretische Informatik},
  %
  % The titles:
  %
  Titel auf Deutsch     = {
    Vorlage für die \LaTeX-Klasse »uzl-thesis« zur Nutzung bei
    Bachelor-~und Masterarbeiten an der  Universität~zu~Lübeck
  }, 
  Titel auf Englisch    = {
    Template for the \LaTeX\ Class “uzl-thesis” for
    Bachelor's and Master's Theses Written at the University~of~Lübeck 
  },
  %
  % Author and supervisor:
  % 
  % Note that the 'Betreuer' or 'Betreuerin' is the supervisor, that
  % is, the professor who officially supervises the thesis. If there
  % is also an assistent of the professor who helped (typically a
  % lot), use 'Mit Unterstützung von' to thank that person. If the
  % thesis was mainly written 'externally' at some company or another
  % institute, point this out using 'Weitere Unterstützung'. 
  % 
  % For your own name, do *not* add things like "BSc" or "BSc
  % cand.". For the supervisor, you should normally include
  % "Prof. Dr." or "PD Dr." (ask your supervisor, what is
  % appropriate), but nothing more (so no
  % "Univ.-Prof. Dr. Dr. h.c. mult." unless your supervisor insists).  
  %
  Autor                 = {Patrick Ugwu},
  Betreuerin            = {Prof. Dr. Javad},
  % 
  % Optional: Supporting persons and institutions. The text should be
  % in German, even for an English thesis.
  %
  Mit Unterstützung von = {Harry Hilfreich},
  % 
  %   Weitere Unterstützung = {
  %     Die Arbeit ist im Rahmen einer Tätigkeit bei der Firma Muster GmbH
  %     entstanden.
  %   },
  %
  %
  % Your Degree Programm (Studiengang)
  %
  % Specify 'Bachelorarbeit' or 'Masterarbeit' and the degree
  % programme. Make sure the name of programme is correct and not
  % some abbreviation or some incorrect variant. For instance:
  % 'Medizinische Ingenierwissenschaft', but not 'MIW';
  % 'Medizinische Informatik', but not 'Medizin-Informatik';
  % 'Informatik', but not 'Informatik (SSE)'.
  %
  % Use German names for German programmes and English names for
  % English ones, so 'Infection Biology', not 'Infektionsbiologie'. 
  % For programmes that have a German bachelor and an English master,
  % use the German name for a bachelor thesis and the English name for
  % the master thesis.
  %
  Bachelorarbeit,
  Studiengang           = {Robotik und autonome Systeme},
  %
  % Date on which the thesis is turned in German, formatted the
  % traditional German way:
  %
  Datum                 = {1. Januar 2024},
  %
  % The English abstract. You must always provide abstracts in German
  % and in English. 
  %
  Abstract              = {
    It is
  },
  Zusammenfassung       = {
    Es ist nicht leicht  
  },
  %
  % Optional: 'Danksagungen' (German) or 'Acknowledgements'
  % (English). Both keys are optional and both have the same effect of
  % adding an acknowledgements text after the abstracts and before the
  % table of contents.
  %
  Acknowledgements      = {
    This is the place 
  },
  % Bibliography style: Choose between
  % 
  % 'Alphabetische Bibliographie'
  % for all degree programmes in the natural sciences 
  % 
  % 'Numerische Bibliographie'
  % alternative for all other degree programmes
  % 
  % Either will load biblatex and setup the citation methods and the
  % bibliography styles correctly. You should not mess with them.
  % 
  Alphabetische Bibliographie,
  % Alternatively:
  % Numerische Bibliographie
}




%%%%%%%%%%%%%%%%%%%%
%
% Styling the thesis
%
%%%%%%%%%%%%%%%%%%%%
%
% Creating a visually pleasing layout and choosing fonts is not
% easy. Furthermore, different people have different preferences. Of
% course, for the University of Lübeck, the dean of studies could just
% force everyone to use one specific layout and font, but that seems a
% bit drastic and, also, it seems nice that thesis by different people
% have an individual style even though they all stick to the same
% overall structure.
%
% For these reasons, I (Till Tantau) have spend quite some time on
% designing a flexible layout and styling mechanism for theses.
%
% Basically, the overall structure of the thesis is fixed by the
% thesis class and so are many structural elements. For instance, you
% cannot change the order in which the abstract and table of contents
% are shown, you cannot move the bibliography elsewhere, indeed, the
% bibliography style is also fixed. Likewise, the text on the title
% page is fixed.
%
% Although many things are fixed, you *can* change several other
% things. For instance, you can change the font used for the main
% text, you can change which font is used for titles and headings or
% you can change whether titles and headlines are centered or flushed
% left.
%
% There are many LaTeX packages for changing such things. You are
% kindly asked *not to use them*. Rather, use (only) the options
% offered by the thesis class. All possible choices and combinations
% there have been tested by me and produce nice results; what happens
% with other packages no one knows and might no longer conform to what
% is expected by the university. As you will see, you still have a
% lot of options.
%
%
% Technical note: All styling is done via the command
%
% \UzLStyle{...}
%
% where ... is a key-value list just as for \UzLThesisSetup. The
% difference is just that everything having to do with styling as
% controlled by \UzLStyle, while the more “formal” setup keys are
% controlled by \UzLThesisSetup.
%
%%%
%
% Designs
%
%
% A \emph{design} is a whole set of font and layout options bundled
% together. They have been chosen in such a way that a visually
% pleasing “overall appearance” results.
%
%
% \UzLStyle{computer modern oldschool design}
%
% The look of this design mimics the “classical” way a paper or report
% created with \LaTeX\ looks like: The Computer Modern font is used,
% bold face fonts are used for headlines, only black and white are
% used as colors. This design reminds me of older scientific
% documents, especially from the computer science community where
% \LaTeX\ was used very early.
%
%
% \UzLStyle{computer modern basic design}
%
% A slightly less “oldschool” version of the previous design. It is
% still a classic design in the sense that it uses the Computer Modern
% font and that it still has this “good old \LaTeX” look, but some
% more modern aspects (like colors!) have been added.
%
% Note that this design uses Myriad for the title page (one of the
% “modern aspect”), which means that his font must be installed.
%
%
% \UzLStyle{computer modern scholary design}
%
% In my opinion, this is the ultimate “scholary design”: The thesis
% will look like it has been typeset by hand some 150 years ago and
% then printed by a university press. There is really nothing “modern”
% about it and the word in the name of the design is just part of the
% name of the “Computer Modern” font.
%
%
% \UzLStyle{pagella basic design}
%
% A, well, basic design that uses the Pagella font rather than the
% Computer Modern font. Especially the bold face version of this font
% looks nicer than the Computer Modern counterpart. Also, Pagella,
% while still having a “bookish” look, still feels a bit fresher than
% Computer Modern. 
%
%
% \UzLStyle{pagella centered design}
%
% A variant of the basic Pagella design that centers all
% headlines. A nice alternative to the basic version.
%
%
% \UzLStyle{pagella contrast design}
%
% This design tries to create some visual friction by contrasting the
% sans serif headline font (in bold!) with the main text. I find it a
% visually very interesting combination.
%
%
% \UzLStyle{alegrya basic design}
%
% The third variant of the basic design, this time using the Alegrya
% font. 
%
%
% \UzLStyle{alegrya scholary design}
%
% The Alegrya version of the previous “scholary” design. Unlike the
% Computer Modern version, this design does not look old, but more
% fresh -- while still creating the impression that the text must be
% about a very scientific subject. 
%
%
% \UzLStyle{alegrya stylish design}
%
% The design is quite similar to the scholary version for the Alegrya
% font, but with even more modern additions. “Stylish” is the word
% that comes to my mind.
%
%
\UzLStyle{alegrya modern design}
%
% A design that uses the sans serif version of the Alegrya font for
% the headlines. This is a nice modern overall design.
%
%%%




%%%%%%%%
%
% Now, include the package you need here using \usepackage. 
%
% However, many standard packages are already loaded by the class:
%
% amsmath, amssymb, amsthm, babel, biblatex, csquotes, etoolbox,
% filecontents, fontspec, geometry, hyperref, tikz (with libraries
% arrows.meta, positioning and shapes), varioref, url 
%
% Indeed, in many cases you will not need any extra packages.
%
%%%%%%%





\begin{document}

%
% The title page and table of contents will be inserted automatically
% here. 
%


\chapter{Einleitung}
% In a German thesis write: \chapter{Einleitung}


% !!!!!!!!!!!!!!!!!!!!!!!!!!!!!!!!!!
% !!! Your action is needed here !!!
% !!!!!!!!!!!!!!!!!!!!!!!!!!!!!!!!!!
%
% Replace with your own introduction:




  - Einführung in das Thema der Schwarmrobotik und die Bedeutung von Simulationen
      
      schwarmrobotik (und HRI?)
        kurze Erklärung
          was versteht man unter: (schwarm)Robotik, Schwarm, collab,.... kurz!
          siehe ba preprint..
        brücke zu simulatoren
      
      
      bedeutung von Simulatoren (gerade in robo, in 2.)):
        preis ersparnis  testumgebung(underwater), 
        in 2.)verhindern von schäden durch fehlerhafte software/algos
        variationen beim testen, mit verschiedenster hardware  
        quellen erwähnen [],[],...

    
      collab überall, im herz der forschung + nutzer? 
      Kurze Erklärung der Bedeutung von kollaborativem Arbeiten und herkömmlichen Simulatoren in diesem Kontext
        keine collab sim gefunden, nicht untersucht
      collab besoonders relevant für schwärme?



      \begin{Code}

      Warum dieses thema? + problem/frage?:
      schwarm bietet sich für collab an  (old papers [?], schhwierigkeiten der HCI -> control of swarm)
      warum ist/könnte kollab in robosim nützlich sein -> kollab vorteile

      relevanz:
      trends: günstigere Technik -> ermöglicht Robotik -> Schwarm -> collab
      pros von sim: ..
      stand der Simulatoren : nicht effizent + flexibel, keine collab

      [3,9,..]

      was will ich lösen/zeigen?
      tool to combine collab, flex and effi:
      argos bringt flex und effi,
      ich bringe collab

      hilft colllab?


      \end{Code}

      


      
   





%\section{Motivation}


%\begin{description}
%  \item[Warum dieses thema? + problem/frage?] 
%  schwarm bietet sich für collab an\\   (old papers [?], schhwierigkeiten der HCI -> control of swarm)
%  
%  \item[relevanz]
%  \begin{itemize}
%    \item trends: günstigere Technik -> ermöglicht Robotik -> Schwarm -> collab
%    \item pros von sim: ..
%    \item stand der Simulatoren : nicht effizent + flexibel, keine collab
%  \end{itemize}
%  
%
%  \item[was will ich lösen/zeigen?]
%  \begin{itemize}
%    \item tool to combine collab, flex and effi:\\  argos bringt flex und effi, \\  ich bringe collab
%    \item hilft colllab?
%  \end{itemize}
%
%\end{description}
%[3,9]




  
    



  

  

  





\section{Beiträge dieser Arbeit}
% In a German thesis write: \section{Beiträge dieser Arbeit}

% !!!!!!!!!!!!!!!!!!!!!!!!!!!!!!!!!!
% !!! Your action is needed here !!!
% !!!!!!!!!!!!!!!!!!!!!!!!!!!!!!!!!!
%
% Replace with a detailed account of your contributions:

- Formulierung der Forschungsfrage und Zielsetzung deiner Arbeit

      Forschungsfrag: ist collab in sim sinnvoll/potenzial

      Ziel: frage beantworten oder zumindest tendenz zu erschließen
        überprüfen, ob collab bei sim sinnvoll/ potenzial durch nutzertest


Beiträge Ihrer Arbeit zur Forschung oder zum Fachgebiet 
Hier können Sie auf die neuen Erkenntnisse, methodologischen Innovationen oder praktischen Implikationen Ihrer Arbeit 
eingehen und damit verdeutlichen, welchen Wert Ihre Forschung für die Fachgemeinschaft hat.


\section{Related Work}
% In a German thesis write: \section{Verwandte Arbeiten}

% !!!!!!!!!!!!!!!!!!!!!!!!!!!!!!!!!!
% !!! Your action is needed here !!!
% !!!!!!!!!!!!!!!!!!!!!!!!!!!!!!!!!!
%
% Replace with a detailed account of what other people have already
% researched concerning your thesis's theme. Even when (indeed,
% especially when) there has been only little or even no research by
% other people, you should explain in detail that this is the case and
% why it is the case. 

Grundlagen der Schwarmrobotik und Simulation 

- Zusammenfassung der relevanten Literatur und früheren Arbeiten zum Thema

collab trend[7]
    The growing trend towards tools that enable collaboration [andere quellen]
    
    Since around 2005, interest in collaborative editors has again increased sharply with the advent of Web 2.0, and the first web-based tools appeared (e.g., Wikis and Blogs). Collaborative solutions that allowed shared editing in real-time and were based on web technologies gradually became widespread. Such tools gained a lot of popularity as many of them were free to use
    
    is a broad field of research
    is a hot topic in research and in practice

    bedeutung von robos:[3]
		
		prob:Modern farms are expected to produce more yields with higher quality at lower expenses in a sustainable way that is less dependent on the labor force. Implementation of digital farming and site-specific precision management are some of the possible responses to this expectation
		
		For example,  instantaneous yield monitoring and estimating the required time for harvesting operation is a labor intensive task, ignored, or is carried out manually
		Currently, there are no reports of a commercial robotic platform that can simultaneously map the yield parameters on-the-go prior to harvesting
		ecoming a critical problem with increasing the uncertainties about the future availability of the labor force that is willing to accept tedious jobs in the harsh greenhouse condition
		manual data sampling implies high costs and low accuracy and is significantly influenced by the interpretation of the person involved
		
		lsg:The functionality of robots when combined with data processing, analyzing models, and artificial intelligence will assist farmers to manage their fields and plants more efficiently


    [6]sim history:
      its development has been tightly related to the developments in computing
      The massification of personal computing during the 90s contributed to the development of several publicly available software tools for robotic manipulator 
	modeling and simulation. Some of these tools included: XAnimate [8], the Robotics Toolbox for Matlab [9], and the Robotica package for Mathematica [10]. 
      During the late 90s, the focus of attention shifted towards manipulation, grasping and mobile robots. 
	This trend led to the development of GraspIt! [5], a simulator for robotic grasping, and numerous tools for simulating and interfacing with various mobile robot platforms. 
	The list of programming libraries, simulation tools and packages can be quite long, however, among those open-source tools that have achieved some degree of sustained development it is 	possible to mention Carmen [14],ODE [19], OpenRave [22], OpenSim [23], the Player-Stage-Gazebo (PSG) project [16], UsarSim [25]. 

  bedeutung von sim in robo:
	[2]computer simulation hat sich bewiesen, anhand
		besonders für mobile autonome robos (inkludiert schwarm)
		1. robos sind zerbrechlich,potenziel gefährlich, schwer zu testen + entwickel (because of their complexity)
		fähigkeit vorher software zu testen minimiert(mitigate) die gefahren und kosten
		2. development + research of software possiable without the hardware/robots
			+ robots come from offsite, at large expense??
		3. ability to test robotic software in situations that are not feasible to create in reality
		4. ability to run hundreds or thousands of trials for the purposes of learning robot control algorithms, crucial if hand coding is difficult
	
	[6] bedeutung von sim in robo
      simulation has been important since the beginning of robotics,
      essential tool in teaching, research, and development  (mit extra quellen => proven)
      to design and test the feasibility of complex systems
      explore design options, 
      
      (use cases:)
      assist in off-line programming of robot systems ??,
       evaluate the performance of robotic cells,
      develop virtual environments for operator training or teleoperation, 
      nor test advanced robot control algorithms .
      
      The role of simulation is also important as teaching and learning the fundamental physics and robotic fundermentals, very hard to grasp without actual experimentation and visualization. 
      
      can be costly => an excellent alternative of doing experiments and testing systems away from the dangers and unpredictability of the physical world.
	
      bedeutung von sim:[3]
      prob:Improvement of robotics for agricultural application however requires experimenting, as well as evaluating, for finding the optimum solution. Experimenting with the physical robots and sensors in an actual field, is not always possible due to the time constraints, unavailability of equipment, and the operation costs
      
      lsg:To accelerate this pace, simulation methods can provide an affordable framework for experimenting with different sensing and acting mechanisms in order to verify the performance functionality of the robot in different scenarios
  

	[10]example?: sim is huge in [10]:
        Experimentation with underwater robots is normally very difficult due to the high number of resources required.
          A water tank –high enough for the systems to be tested– is normally needed, 
            which requires significant space and maintenance. 
          Another possibility is to access to open environments such as lakes or the sea, but this normally involves high costs and requires special logistics. 
          
          very difficult for researchers (operating in the surface) to observe the evolution of the running system



essential sim qualities(ideal sim)[2]:
	accurate physics
	high quality rendering
	open source
	multi platform
	easy with ros


  daher brauch es sim, der:[10]
  (i) develop and test the systems before they are deployed, and 
  (ii) supervise a real underwater task where the developers do not have a direct view of the system
bisher meist  projekt spezifisch, obsolet oder commercial


[6]sim problems/challenges:
      Researchers, developers and educators have a hard time to decide which is the right tool for mobile robot simulation -> solved by surveys
      some sim have this, some that -> impliziert das keiner alles hat







- Identifizierung von Forschungslücken oder Diskussionen in der Literatur   
collab


- Beschreibung verschiedener Simulationsansätze und ihrer Bedeutung für die Robotik
    ansätze?: spiel - spezific - general - modular ??
    geschichte der sim- aus spiel zu eigenen sim, aus projektspezifisch zu general


- Überblick über die Grundkonzepte der Schwarmrobotik und deren Anwendungen
    noch relevanter in zukunft


- Diskussion über die Herausforderungen und Vorteile von Simulationen in der Schwarmrobotik
    herausforderungen : user feedback paper, accuracy, performance, ..
    Vorteile: unbekannt, nicht untersucht ?


\section{Aufbau dieser Arbeit}
% In a German thesis write: \section{Aufbau dieser Arbeit}


% !!!!!!!!!!!!!!!!!!!!!!!!!!!!!!!!!!
% !!! Your action is needed here !!!
% !!!!!!!!!!!!!!!!!!!!!!!!!!!!!!!!!!
%
% Replace the following by one or two paragraphs describing the
% thesis's structure.

structure 

1 introduction
2. mein sim
3. user test
4. Ergebnisse + evaluation
5 con
6 bibliography









% !!!!!!!!!!!!!!!!!!!!!!!!!!!!!!!!!!
% !!! Your action is needed here !!!
% !!!!!!!!!!!!!!!!!!!!!!!!!!!!!!!!!!
%
% Replace the whole text chapter with the main text of your thesis! 

\chapter{meinsim}%: Document Setup and Document Structure}
\label{chapter-use}



   
   --reroll-
   
   Ja, das ändert definitiv etwas! Bei einem Thema wie der Untersuchung von kollaborativem Arbeiten an einer Simulation 
   für Schwarmrobotik im Vergleich zu einem herkömmlichen Simulator könnten einige spezifische Abschnitte und Schwerpunkte 
   relevant sein. Hier ist eine angepasste Struktur mit einer möglichen Gewichtung:

1. **Einleitung (ca. 10%):**
   - Einführung in das Thema der Schwarmrobotik und die Bedeutung von Simulationen
   - Problemstellung und Zielsetzung der Untersuchung
   - Kurze Erklärung der Bedeutung von kollaborativem Arbeiten und herkömmlichen Simulatoren in diesem Kontext

2. **Literaturüberblick (ca. 20%):**
   - Übersicht über relevante Arbeiten zur Schwarmrobotik und Simulationstechniken
   - Diskussion über die Vor- und Nachteile von kollaborativem Arbeiten und herkömmlichen Simulatoren
   - Identifizierung von Forschungslücken in Bezug auf die direkte Vergleichbarkeit von kollaborativem Arbeiten und herkömmlichen Simulatoren in der Schwarmrobotik

3. **Methodik (ca. 15%):**
   - Beschreibung der Simulationen und Tools, die für die Studie verwendet wurden
   - Erläuterung der Parameter und Metriken, die zur Bewertung von kollaborativem Arbeiten und herkömmlichen Simulatoren verwendet wurden
   - Details zur Durchführung der Experimente und zur Datenerfassung

4. **Ergebnisse (ca. 25%):**
   - Präsentation der quantitativen und qualitativen Ergebnisse der Simulationen
   - Vergleich der Leistung von kollaborativem Arbeiten und herkömmlichen Simulatoren anhand der definierten Metriken
   - Grafische Darstellung von Daten und statistische Analysen

5. **Diskussion (ca. 25%):**
   - Interpretation der Ergebnisse im Hinblick auf die Forschungsfragen und Hypothesen
   - Diskussion über die praktischen Implikationen der Ergebnisse für die Schwarmrobotik-Forschung und -Entwicklung
   - Reflexion über die Stärken und Schwächen der durchgeführten Studie und mögliche Verbesserungen

6. **Schlussfolgerung und Ausblick (ca. 5%):**
   - Zusammenfassung der wichtigsten Erkenntnisse und Schlussfolgerungen
   - Ausblick auf zukünftige Forschungsrichtungen, die sich aus den Ergebnissen ergeben, und potenzielle Anwendungen von kollaborativem Arbeiten in der Schwarmrobotik

Diese Struktur berücksichtigt die spezifischen Anforderungen und Besonderheiten deines Themas 





\chapter{user test} 
   
   3. **Vergleich von kollaborativem Arbeiten an einer Simulation und einem herkömmlichen Simulator (ca. 30%):**
      - Darstellung der Funktionsweise und Merkmale beider Simulationsansätze
      - Analyse von Fallstudien oder Experimenten, die kollaboratives Arbeiten und herkömmliche Simulationen vergleichen
      - Bewertung der Vor- und Nachteile jedes Ansatzes im Kontext der Schwarmrobotik
   
   4. **Methodik (ca. 15%):**
      - Beschreibung der Methoden, die du zur Durchführung deiner Untersuchung verwendet hast
        task in beiden umgebungen + survey
      
        - Erläuterung deiner Herangehensweise und Begründung der Methodenwahl 
      
      - Erklärung der Kriterien und Metriken, die zur Bewertung der Simulationsansätze herangezogen wurden

      - Erläuterung des experimentellen Aufbaus und der Durchführung der Vergleichsstudie
      task:
      wer waren Versuchpersonen
      erfasste Daten:


      \begin{Code}
        Wie leicht war es, mit anderen Freiwilligen in der Simulation zusammenzuarbeiten?
        Welche Aspekte der Simulation haben die Kollaboration unterstützt oder behindert?
        Konnten Sie effektiv mit der Simulationsumgebung interagieren, um die gestellten Aufgaben zu lösen?
        verbesserungsvorschläge?

        -------------------2.entwurf-------------------------
        task: 
        platzieren von hindernissen an bestimmten stellen, -> customisible environment
        platzieren und anpassen von robos -> customisible robot models
        robo verhalten erweitern (aus verhalten einen aspekt rauslöschen?) -> testing of control algos
        
        Qualitative results - based on responses received from the surveys. 
          These give a picture of user experience and preference. 
            
            pre
              How often do you use computers? 
              How would you rate your knowlage about (swarm) robotics?
              How would you rate your experience in programming robots or robot swarms?
              How would you rate your experience in simulating robots or robot swarms?
          
          
            post
              ease
              Which simulation environment/interface was easier to use? - easier
              Which simulation environment would you prefer for more frequent use? - Workload  , usable vs ease
              7. I would imagine that most people would learn to use this system very quickly -ease
      
              usability
              How do you rate the clarity and usability of the simulation? - Usability
              How do you rate the clarity and usability of the collaboration functions? - Usability, collab gut umgeetzt?
              8. I found the system very cumbersome to use - implementation of features or just argos???
              Were you able to form a accurate projection from your mental picture of the environment? - Accuracy
              5. I found the various functions in this system were well integrated - woran hats gelegen (falls nicht gut)
              
              collab
              2. I found the system/collab functions unnecessarily complex -> war collab unnötig
              9. I felt very confident using the system - usable/easy to understand, well integrated features
              10. I needed to learn a lot of things before I could get going with this system - no robo exp or not usable/understandable
            
              
              How would u rate each simulation? - overall + for comparission -> success?
        
      
        Quantitative results - from robot trajectory and related information 
          These serve as performance metrics that tell us whether there is an actual difference in how well 
          the participants performed, between interfaces. 
            recording task completion  
            recording task completion time
            (clicks, how many clicks -> much looking around/wondering -> lacking clarity???)
            .
            .
            .
            and other related data
      
      
      #################------metriken-----------############
        og
        Qualitative results - based on responses received from the surveys. 
          These give a picture of user experience and preference. 
        Quantitative results - from robot trajectory and related information 
          (recording task completion time, swarm trajectory, collision count, and other related data)
          These serve as performance metrics that tell us whether there is an actual difference in how well 
          the participants performed, between interfaces. 
          
      
      
        in ba
        Metrics from two categories will be applied in the evaluation. 
          The Performance category includes metrics such as effectiveness, accuracy, and response time, 
            which have a measurable impact on the usability and performance of the respective HSI. 
          The User Experience category includes metrics such as user-friendliness, immersion, and workload, 
            which represent a subjective assessment by the participants and have an indirect influence on performance [12].
      ######################-#-------#-#--#-#-############## 
      
      
      
      
      ------------------1.entwurf---------------
      task: Entwirf einen Raum bestehend aus 3 "Hindernisen" und einem schwarm von 10 (identischen) "Robotern" in Argos und Argos+
      
      post
        Which simulation environment/interface was easier to use?
        Which simulation environment would you prefer for more frequent use? - Workload  
        How do you rate the clarity and usability of the simulation? - Usability
        How do you rate the clarity and usability of the collaboration functions? - Usability
        Were you able to form a accurate projection from your mental picture of the environment? - Accuracy
        
        1. I think that I would like to use this system frequently - usable + ease
        2. I found the system unnecessarily complex -> war collab unnötig
        5. I found the various functions in this system were well integrated - woran hats gelegen (falls nicht gut)
        7. I would imagine that most people would learn to use this system very quickly -ease
        8. I found the system very cumbersome to use - implementation of features or just argos???
        9. I felt very confident using the system - usable/easy to understand/well integrated features
        10. I needed to learn a lot of things before I could get going with this system - no robo exp or not usable/understandable
      
        How would u rate each simulation?
      
      
      
        ------------------literaturr----------------------------------
      
          pre
            experience  
            computer
            1. How often do you use computers? 
      
            robot interaction ?
            2. How often do you interact with robots?
            3. How often do you interact with robot swarms or multi-robot systems? 
            
            robot simulation
            4. How would you rate your experience in simulating robots? 
            5. How would you rate your experience in simulating robot swarms? 
      
      
          post
            in og
            1. Workspace Awareness 
            2. Robot Awareness 
            3. Ease of Use 
            4. Response accuracy 
            5. Overall Preference 
            6. Overall Average Rating 
      
      
            in ba
            8. Which interface gave you a better sense of the mission environment? - Immersion
            9. Which interface gave you a better sense of the robots? - Immersion
            10. Which interface was easier to use? - Usability
            11. Which interface was more accurate in processing your inputs? - Accuracy
            12. Which interface would you prefer for more frequent use? - Workload
      
      
            in ba, woher?
              1. Did you feel like part of the swarm? - Immersion
              2. Were you aware of the swarm’s current goal at all times? - Usability
              3. Were you able to form a mental picture of the situation based on the information provided? - Immersion
              4. How do you evaluate the quality of communication regarding the information provided by the swarm? - Usability
              5. How do you evaluate the quality of communication regarding the instructions you provided to the swarm? - Accuracy
              6. Were your commands executed in a timely manner or within a relevant timeframe? - Response Time
            
              7. How do you rate the clarity and usability of the simulation and collaboration functions? - Usability
            
            
      
      
            SUS - System Usability Scale
            1. I think that I would like to use this system frequently
            2. I found the system unnecessarily complex
            3. I thought the system was easy to use
            4. I think that I would need the support of a technical person to be able to use this system
            5. I found the various functions in this system were well integrated
            6. I thought there was too much inconsistency in this system
            7. I would imagine that most people would learn to use this system very quickly
            8. I found the system very cumbersome to use
            9. I felt very confident using the system
            10. I needed to learn a lot of things before I could get going with this system 
      
      \end{Code}


    
\chapter{Ergebnisse und Diskussion}
      4. **Ergebnisse (ca. 25%):**
      - Präsentation und Diskussion deiner Forschungsergebnisse
      - Grafische Darstellung von Daten, wenn nötig
      - Interpretation der Ergebnisse im Hinblick auf deine Forschungsfrage(n)
   
   5. **Diskussion (ca. 25%):**
      - Interpretation deiner Ergebnisse im Kontext der vorhandenen Literatur
      - Reflexion über mögliche Einschränkungen deiner Studie
      - Vorschläge für zukünftige Forschung
      
      5. **Ergebnisse und Diskussion (ca. 25%):**
         - Präsentation der Ergebnisse deiner Untersuchung, einschließlich quantitativer Daten und qualitativer Beobachtungen
         - Interpretation der Ergebnisse im Hinblick auf deine Forschungsfrage und Hypothesen
         - Diskussion der Implikationen deiner Ergebnisse für die Schwarmrobotik und mögliche zukünftige Entwicklungen


\chapter{Zusammenfassung und Ausblick}
% In a German thesis write: \subsection{Zusammenfassung und Ausblick}


% !!!!!!!!!!!!!!!!!!!!!!!!!!!!!!!!!!
% !!! Your action is needed here !!!
% !!!!!!!!!!!!!!!!!!!!!!!!!!!!!!!!!!
%
% Replace the following with your conclusion





6. **Schlussfolgerung und Ausblick (ca. 5%):**
   - Zusammenfassung der wichtigsten ergebnisse und Erkenntnisse deiner Arbeit/Schlussfolgerungen, die sich aus deiner Arbeit ergeben
   lässt sich tendenz erkennen, was sagen die ergebnisse aus

   - Bewertung der Bedeutung deiner Forschungsergebnisse für die Robotik und Simulation
   urprungsfrage beantworten
   
   - Ausblick auf potenzielle weitere Forschungsbereiche und Anwendungsmöglichkeiten/- Ausblick auf mögliche Implikationen und Anwendungsbereiche deiner Forschung
   erweiterte funktionen?
   in allen sim sinnvoll?



\chapter{tex semantik}

\section{Section 1}
section 1


\section{section2}
section 2


Some code:
\begin{Code}
print(1+1=3)
\end{Code}

Hier kommt \\ ein Zeilenumbruch

   eine Aufzählung:

   \begin{enumerate}
   \item 
     first
   \item
     second...
   \end{enumerate}
   
   andere Aufzählung:
   
   \begin{description}
   \item[Stier] 
     ist ein tier
   \item[stein]
     ist kein tier
   \end{description}
   
   noch eine aüfzählung:
   
   \begin{itemize}
     \item number one....
     \item
       number two...
     \end{itemize}
   
   
   \subsection{Subsection 1}
   ....
   Figure
   
   \begin{figure}[htbp]
     \centering
     \textcolor{black!10}{\vrule width3cm height 3cm}
     \caption{Figure Discription}
     \label{fig-ode1}
   
   \end{figure}
   
   
   \subsection{Subsection 2}
   ...
   
   For tables:
   
   
   \begin{table}[htpb]
     \caption{Table description}
     \label{fig-tab1}
     \centering
     \begin{tabular}{lp{5cm}}
       \uzlhline
       \uzlemph{Animal} & \uzlemph{Sound} \\ \uzlhline
       Cat & Meow \\
       Dog & Wuff or bark\\ \uzlhline
     \end{tabular}
   \end{table}




% Normally, the bibliography comes next at this point. Do *not* (try
% to) include further indices and tables like an index or
% a list of figures or a list of tables or such things. Nobody
% actually uses them and they just use up space. 
%
% You *can* however include a glossary, if this seems appropriate. It
% goes here as an unnumbered chapter. Most thesis will *not* need a
% glossary: a well-written text (re)explains strange words and
% concepts as necessary. However, there are situations where a
% glossary may be helpful.








%%%
% 
% Bibliographies
%
%%%
%
% The uzl-thesis class will load biblatex for the bibliography
% management. This is a powerful package, see its documentation for
% details. The styles will be setup correctly and automatically by
% choosing one of the two style keys as described earlier.
%
% In order for the bibliography to work, run latex in the following
% order (which is the standard order):
% 
% > lualatex thesis-example
% > bibtex thesis-example
% > lualatex thesis-example
% 
% Add BibTeX files using \addbibresource or use the {bibtex entries}
% environment (see below).
%
%%%
%
% Although everyting is normally setup automatically, you can change
% the options passed to biblatex using the key 'biblatex';
% for instance,
%
%   \UzLThesisSetup{biblatex={firstinits=false}}
%
% will switch off shortened first names. Normally, you will not need
% this key in your preamble. 
% 
% Note that the bibtex program is used as the 'backend' of biblatex
% by default (rather than biber, which is the preferred program of
% biblatex). This means that you can (and must) run *bibtex* after you
% have run lualatex on your thesis. If you wish to use biber instead
% of bibtex, say 'biblatex={backend=biber}'. 
% 
%%%
%
% The following environment is optional. It allows you to keep the
% bibtex entries for your thesis right here in the thesis file. What
% happens is that each time this tex file is processed, the contents
% of the following environment gets written to the file
% \jobname-bibtex-entries.bib (this file gets overwritten each
% time). Independently, \addbibresource{\jobname-bibtex-entries.bib}
% is always called if the file \jobname-bibtex-entries.bib
% exists. 
%
% In result, you can edit and keep the bibliography's bibtex entries
% right here. If you change something here, run latex, then bibtex,
% then latex once more.
%
% If you would like to manage the bibtex entries in a separate file,
% remove the below environment, delete the \jobname-bibtex-entries.bib
% file and instead write
%
% \addbibresource{filename-of-your-bibtex-file.bib}
%
% in the preamble.
%
%%%


% !!!!!!!!!!!!!!!!!!!!!!!!!!!!!!!!!!
% !!! Your action is needed here !!!
% !!!!!!!!!!!!!!!!!!!!!!!!!!!!!!!!!!
%
% Replace following example entries with the ones of your thesis.

\begin{bibtex-entries}

@Book{Knuth1986,
  author =       {Donald Erwin Knuth},
  title =        {The \TeX book},
  publisher =    {Addison-Wesley},
  year =         {1986},
}

@Book{Lamport1994,
  author =       {Leslie Lamport},
  title =        {\LaTeX: A Document Preparation System},
  publisher =    {Addison-Wesley},
  edition =      {Second edition},
  year =         {1994},
}

@TechReport{Kernighan1974,
  author =       {Brian Kernighan},
  title =        {Programming in C – A Tutorial},
  institution =  {Bell Laboratories},
  year =         {1974}
}

@Manual{Tantau2019,
  author =       {Till Tantau},
  title =        {The Ti\emph kZ and PGF Packages: Manual for version 3.1.3},
  institution =  {Institut für Theoretische Informatik, Universität zu Lübeck},
  year =         {2019},
  url =          {https://github.com/pgf-tikz/pgf}
}

@Book{Alley1996,
  author =       {Michael Alley},
  title =        {The Craft of Scientific Writing},
  publisher =    {Springer},
  year =         {1996},
  edition =      {Third Edition},
}

@Book{DowneyF13,
  author =       {R. G. Downey and M. R. Fellows},
  title =        {Fundamentals of Parameterized Complexity},
  series =       {Texts in Computer Science},
  publisher =    {Springer},
  year =         2013,
  doi =          {10.1007/978-1-4471-5559-1},
}

@Manual{biblatex,
  title =        {The \textsc{BibLaTeX} package},
  subtitle =     {Sophisticated Bibliographies in \LaTeX},
  author =       {Kime, Philip and Lehman, Philipp},
  url =          {https://github.com/plk/biblatex},
  urldate =      {2019-06-11},
  date =         {2018-10-30},
  version =      {3.12}
}

@Manual{varioref,
  title =        {The \textsc{varioref} package},
  subtitle =     {Intelligent page references},
  author =       {Mittelbach, Frank},
  url =          {http://www.ctan.org/pkg/varioref},
  urldate =      {2019-06-11},
  date =         {2016-02-16},
  version =      {1.5c}
}

@Manual{hyperref,
  title =        {The \textsc{hyperref} package},
  subtitle =     {Extensive support for hypertext in \LaTeX},
  author =       {Rahtz, Sebastian and Oberdiek, Heiko},
  url =          {https://github.com/ho-tex/hyperref},
  urldate =      {2019-06-11},
  date =         {2018-11-30},
  version =      {6.88e}
}

@Manual{babel,
  title =        {The \textsc{babel} package},
  subtitle =     {Multilingual support for Plain \TeX\ or \LaTeX},
  author =       {Braams, Johannes L. and Bezos López, Javier},
  url =          {http://www.ctan.org/pkg/babel},
  urldate =      {2019-06-11},
  date =         {2019-06-03},
  version =      {3.32}
}

@Manual{fontspec,
  title =        {The \textsc{fontspec} package},
  subtitle =     {Advanced font selection in Xe\LaTeX\ and Lua\LaTeX},
  author =       {Robertson, Will},
  url =          {http://www.ctan.org/pkg/fontspec},
  urldate =      {2019-06-11},
  version =      {2.7c}
}

@Manual{url,
  title =        {The \textsc{url} package},
  subtitle =     {Verbatim with \textsc{url}-sensitive line breaks},
  author =       {Arseneau, Donald},
  url =          {http://www.ctan.org/pkg/url},
  urldate =      {2019-06-11},
  date =         {2013-09-16},
  version =      {3.4}
}

@Manual{amsmath,
  title =        {The \textsc{amsmath} package},
  subtitle =     {\AmS\ mathematical facilities for \LaTeX},
  author =       {{The \LaTeX\ Team}},
  url =          {http://www.ams.org/tex/amslatex.html},
  urldate =      {2019-06-11}, 
  date =         {2017-09-02},
  version =      {2.17a}
}

@Book{Beutelspacher2009,
  title =        {„Das ist o.\,B.\,d.\,A.\ trivial!“: Tipps und Tricks zur
                  Formulierung mathematischer Gedanken (Mathematik für
                  Studienanfänger)},
  author =       {Albrecht Beutelspacher},
  year =         {2009},
  edition =      {Ninth, updated edition},
  publisher =    {Vieweg+Teubner Verlag},
  doi =          {10.1007/978-3-8348-9075-7},
}

\end{bibtex-entries}



% If you need to have an appendix (I advise against it), insert it
% here using, first, \appendix and then \chapter and then,
% possibly, \section. 
%
% \appendix
%
% \chapter{Technical Appendix}
%
% \section{Experimental Parameters} % possibly
%
% Again, I advise against using an appendix.


\end{document}

%  LocalWords:  LaTeX tex moretexcs Lübeck pdf uzl lualatex bibtex th
%  LocalWords:  TechReport Kernighan Lamport's Tantau's Tantau cls kZ
%  LocalWords:  Mustermann emacs oldschool pdflatex texmf utf biber
%  LocalWords:  biblatex Alphabetische Bibliographie Numerische VIIa
%  LocalWords:  varioref german Einleitung Beiträge dieser Arbeit xml
%  LocalWords:  Ergebnisse Verwandte Arbeiten Aufbau nucleotide VIIc
%  LocalWords:  ensembl amino phylogenetic Alexa Siri decrypt versa
%  LocalWords:  cryptographic pre nondeterministic deterministically
%  LocalWords:  Beutelspacher Untersuchungen zum genetischen sep llcc
%  LocalWords:  Beispiel tikz jpg png Alegrya Kasimir Malewitsch PGF
%  LocalWords:  Lamport Institut für Theoretische Informatik zu url
%  LocalWords:  Universität Springer DowneyF Downey Parameterized doi
%  LocalWords:  BibLaTeX Kime Philipp urldate Mittelbach hyperref Lua
%  LocalWords:  Rahtz Oberdiek Heiko Braams Bezos López fontspec Das
%  LocalWords:  Arseneau amsmath ist Tipps und zur Formulierung
%  LocalWords:  mathematischer Gedanken Mathematik Studienanfänger
%  LocalWords:  Albrecht Vieweg Teubner Verlag
